%%
%% Copyright (c) 2018 Weitian LI
%% CC BY 4.0 License
%%
%% Resume / Résumé
%% A short document (1-2 pages) to sum up the job-related accomplishments
%% and experience.
%%
%% Information Checklist:
%% * Contact Information
%% * Work History / Experience
%% * Education
%% * Skills
%% * Summary & Objective (optional)
%% * Hobbies & Interests (optional)
%%
%% References:
%% * CV vs. Resume: What is the Difference? When to Use Which?
%%   https://uptowork.com/blog/cv-vs-resume-difference
%% * How to Make a Resume: A Step-by-Step Guide (+30 Examples)
%%   https://uptowork.com/blog/how-to-make-a-resume
%% * Entry-Level Resume: Sample and Complete Guide (+20 Examples)
%%   https://uptowork.com/blog/entry-level-resume-example
%%
%% Weitian LI <liweitianux@live.com>
%% 2018-04-14
%%

% English version
\documentclass{resume}

% File information shown at the footer of the last page
\fileinfo{%
  \faCopyright{} 2018 Weitian LI,
  \creativecommons{by}{4.0}, v\today
}

% Position and place
\newcommand{\myposition}{Data Analyst}
\newcommand{\myplace}{Shenzhen}

\name{Weitian}{LI}
\tagline{\myposition{} @ \myplace}
\taglineicon{\faBinoculars}
\keywords{\myposition, \myplace, Python, Linux, BSD}
\photo{7em}{photo}

% NOTE: adjust the ordering and styles w.r.t. Chinese version
\socialinfo{
  \mobile{132-6262-0332}
  \email{liweitianux@live.com}
  \github{liweitianux} \\
  \degree{Ph.D. (candidate), Physics}
  \university{Shanghai Jiao Tong University (SJTU)} \\
  \address{Shanghai}
  \home{Shaoyang, Hunan}
  \birthday{September 26, 1991}
}

\begin{document}
\makeheader

%======================================================================
% Summary & Objectives
%======================================================================
Highly-motivated Ph.D. candidate in Physics with good foundations of math
and statistics, and familiar with the basics of data analysis, machine
learning, as well as signal and image processing.
Enthusiastic about computer technologies with 10 years experience
in Linux and BSD, and involved in several open source projects (e.g.,
DragonFly BSD).
Skilled in programming (Python, R, etc.) and data analysis,
and looking to fill a position as a \emph{\myposition} at your company
that I can grow with as I achieve company goals.

%======================================================================
% Competences / Skills
\sectionTitle{Competences}{\faWrench}
%======================================================================
\begin{competences}[10em]
  \comptence{Operating Systems}%
    {Linux (10 years), BSD (DragonFly BSD, FreeBSD; 7 years)}
  \comptence{Programming}%
    {Python, Shell, C; R, Julia}
  \comptence{Data Analysis}%
    {R, pandas, scikit-learn; matplotlib, ggplot2; SQL (basic knowledge)}
  \comptence{Tools}%
    {Regular expression; Jupyter notebook; SSH, Git, Make; Ansible}
  \comptence{Web Development}%
    {Django, Tornado; jQuery, Bootstrap; JavaScript, HTML5}
  \comptence{Typesetting}%
    {\LaTeX, Con\TeX{}t}
\end{competences}

%======================================================================
% Education
\sectionTitle{Education}{\faGraduationCap}
%======================================================================
\begin{educations}
  \education%
    {September 2013}%
    {Shanghai Jiao Tong University}%
    {School of Physics and Astronomy}%
    {Physics}%
    {Ph.D. (candidate; expected to graduate in early 2019)}

  \separator{0.5em}
  \education%
    {September 2009}%
    [June 2013]%
    {Shanghai Jiao Tong University}%
    {Department of Physics and Astronomy}%
    {Applied Physics}%
    {Bachelor's Degree}
\end{educations}

%======================================================================
% Research Projects
\sectionTitle{Research Projects}{\faCogs}
%======================================================================
\begin{projects}
  \project
    {January 2015}%
    {National Natural Science Foundation of China}%
    {Key Program}%
    {Simulation of Low-Frequency Radio Sky and Separation of Weak
     Astronomical Signals}%
    {\begin{itemize}
      \item Collaborated in classifying the radio galaxies according to
        the morphologies using a deep Convolutional Neutral Network (CNN).
      \item Developed the
        \link{https://github.com/liweitianux/fg21sim}{\texttt{FG21sim}}
        software to simulate low-frequency radio sky images.
      \item Used algorithms such as wavelet to denoise and enhance X-ray
        astronomical images.
      \item Extracted both the spatial and spectral information of X-ray
        images, and used the Support Vector Machine (SVM) to identify the
        potential point sources.
      \item Significantly improved the modeling of radio halos,
        and integrated the instrumental effects of radio interferometers
        into the simulation pipeline.
      \item Quantitatively evaluated the impacts of radio halos on the
        detection of reionization signals, and finished the journal paper.
    \end{itemize}}%
    [Python, Machine learning, CNN, SVM, Image processing]

  \separator{0.7em}
  \project
    {July 2012}%
    [December 2014]%
    {National Natural Science Foundation of China}%
    {Fund for Distinguished Young Scholars}%
    {The X-ray Study of Galaxies and Clusters of Galaxies,
      and the Research of Cosmic Low-Frequency Radio Radiation}%
    {\begin{itemize}
      \item Reduced the data of over 200 galaxy clusters observed by the
        \textit{Chandra} X-ray Observatory, and analyzed the images and
        spectra.
      \item Built a sample of galaxy clusters, collected optical data from
        SDSS, and investigated the correlation between the central emission
        excess and the central dominating galaxy.
      \item Developed and maintained a suite of data analysis utilities:
        \link{https://github.com/liweitianux/chandra-acis-analysis}%
          {\texttt{chandra-acis-analysis}}.
    \end{itemize}}%
    [Data collection, Data reduction, Statistical analysis, Python, Shell]
\end{projects}

%======================================================================
% Experience
\sectionTitle{Experience}{\faBriefcase}
%======================================================================
\begin{experiences}
  \experience
    {April 2018}%
    {Attended \enquote{The 2$^{\rm nd}$ China-Australia SKA Big Data Workshop.}}%
    [\begin{itemize}
      \item Implemented the data transmission functionality between the
        \link{https://github.com/ICRAR/ngas}{\texttt{NGAS}}
        data storage system and the
        \link{https://github.com/ICRAR/daliuge}{\texttt{DALiuGE}}
        data processing system.
      \item Gained team collaboration experience and
        learned agile development methods.
    \end{itemize}]%
    [Data transmission, Data storage, Agile development, Python]

  \separator{0.5em}
  \experience
    {March 2018}%
    {Became a DragonFly BSD committer.}%
    [][BSD, Open source]

%  \separator{0.5em}
%  \experience
%    {March 2018}%
%    {Used Ansible to manage the VPS configurations,
%      and hosted authoritative DNS service for personal domains.}%
%    [][BSD, Ansible, DNS]

  \separator{0.5em}
  \experience
    {February 2018}%
    {Revised \enquote{The Chinese SKA Science White Book} by
      rewriting the \enquote{Low-Frequency Observation Instruments} section.}

%  \separator{0.5em}
%  \experience
%    {December 2017}%
%    {Participated in configuring and testing the high-performance computing
%      cluster prototype for building the SKA Regional Science Center at
%      Shanghai Astronomy Observatory.}

  \separator{0.5em}
  \experience
    {September 2017}%
    {Involved in writing the
      \enquote{Large-scale Diffuse Foreground Sources} section for
      \enquote{The Chinese SKA Science White Book.}}

  \separator{0.5em}
  \experience
    [April 2017]%
    {August 2017}%
    {Lung CT scan images analysis}%
    [\begin{itemize}
      \item Collaborated with \textit{Shanghai Chest Hospital},
        attempted to identify the mutation types of lung tumors
        by analyzing the CT scan images, in order to formulate a
        better treatment plan.
      \item Extracted the image features using the Gray Level Concurrence
        Matrix (GLCM) and reduced with Principle Component Analysis (PCA),
        but found that the information provided by the CT images was
        insufficient to reliably predict the mutation types.
    \end{itemize}]%
    [Feature extraction, Data reduction, PCA]

%  \separator{0.5em}
%  \experience
%    {April 2017}%
%    {Configured a VPS running DragonFly BSD and serving personal email,
%      website, CalDAV/CardDAV, Git, etc.}%
%    [][BSD, Postfix, Dovecot, Nginx, Radicale, Git]

%  \separator{0.5em}
%  \experience
%    {December 2016}%
%    {Built and administrated a 4-node computer cluster for the team
%      to research the galaxy cluster merger processes by carrying out
%      hydrodynamic simulations.}%
%    [][CentOS, Slurm, Numerical simulation]

%  \separator{0.5em}
%  \experience
%    {November 2016}%
%    {Attended \enquote{BSD Meetup: BSD \& Cloud} @ Shanghai.}%
%    [][BSD, Open source]

  \separator{0.5em}
  \experience
    {September 2016}%
    {Participated
      \enquote{The 13$^{\rm th}$ China Post-Graduate Mathematical
        Contest in Modeling.}}%
    [\begin{itemize}
      \item Learned the Genome-Wide Association Study (GWAS) method
        to locate the most likely Single-Nucleotide Polymorphisms (SNPs)
        associated with a trait or disease.
      \item Used the R programming language to perform Logistic regressions
        and hypothesis testings between SNPs and traits, and
        identified the most possible SNPs and genes that may cause the
        disease.
    \end{itemize}]%
    [R, Data cleansing, Regression analysis, Hypothesis testing]

  \separator{0.5em}
  \experience
    [April 2014]%
    {July 2014}%
    {Organized
      \enquote{The 1$^{\rm st}$ China--New Zealand Joint SKA Summer School.}}%
    [\begin{itemize}
      \item Designed and made the poster.
      \item Designed and developed the website,
        providing functionalities including user registration,
        agenda management, announcements, lecture downloads, etc.
    \end{itemize}]%
    [Design, Django, Bootstrap, jQuery, JavaScript, MySQL]

  \separator{0.5em}
  \experience
    [July 2013]%
    {September 2013}%
    {Summer intern @ 97 Suifang (startup company)}%
    [\begin{itemize}
      \item Developed the website to help patients with \textit{hepatitis B}
        track various indicators in their analysis reports.
      \item Implemented the user registration, data storage and search
        functions in the back end.
      \item Used AJAX in the front end to visualize the temporal variations
        of the indicators.
    \end{itemize}]%
    [Database, Data visualization, Django, AJAX]

%  \separator{0.5em}
%  \experience
%    [March 2010]%
%    {September 2011}%
%    {Involved in the Open Source Association of school}%
%    [][Open source, Linux]
\end{experiences}

%======================================================================
% Languages
%
% References:
% * How should I indicate language proficiency on my resume?
%   https://workplace.stackexchange.com/a/10007
% * How to write resume foreign language skills
%   https://radiantresumeservices.com/how-to-write-resume-foreign-language-skills/
%
\sectionTitle{Languages}{\faLanguage}
%======================================================================
\begin{entries}[|]
  \entry{\textbf{English}}{%
    \begin{description}
      \item[Reading] --- Intermediate
        (read technical documentation and literature)
      \item[Writing] --- Intermediate
        (write journal papers)
      \item[Listening \& Speaking] --- Conversant
    \end{description}
  }

  \separator{0.7em}
  \entry{\textbf{Chinese}}{%
    \begin{description}
      \item[Writing] --- Good
        (involved in writing and revising fund applications,
        annual reports, etc.)
      \item[Speaking] --- Good
        (5 semesters of teaching assistant experience)
    \end{description}
  }
\end{entries}

%%======================================================================
%% Teaching Assistant
%\sectionTitle{Teaching Assistant}{\faUsers}
%%======================================================================
%\begin{entries}
%  \entry{Spring 2017}%
%    {The Universe Around Us (liberal education)}
%  \entry{Fall 2015}%
%    {Introduction to Physics II (Zhiyuan Honors Program)}
%  \entry{Spring 2015}%
%    {Introduction to Physics I (Zhiyuan Honors Program)}
%  \entry{Fall 2014}%
%    {Introduction to Physics I (Zhiyuan Honors Program)}
%  \entry{Spring 2014}%
%    {College Physics (Outstanding Teaching Assistant Award)}
%\end{entries}

%======================================================================
% Awards / Scholarships / Certificates
\sectionTitle{Awards \& Certificates}{\faTrophy}
%======================================================================
\begin{entries}
  \entry{September 2016}%
    {Participation Award,
      The 13$^{\rm th}$ China Post-Graduate Mathematical Contest in Modeling}
  \entry{July 2014}%
    {Outstanding Teaching Assistant, College Physics}
  \entry{November 2013}%
    {Outstanding Ph.D. Student Entrance Scholarship of Shanghai Jiao Tong University}
  \entry{October 2012}%
    {Advanced Individual of Shanghai Jiao Tong University}
  \entry{December 2011}%
    {National Astronomical Observatory Scholarship}
  \entry{September 2011}%
    {Network Engineer (Level 4), National Computer Rank Examination}
\end{entries}

\end{document}

%% EOF
