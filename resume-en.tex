%%
%% Copyright (c) 2018-2019 Weitian LI <wt@liwt.net>
%% CC BY 4.0 License
%%
%% Résumé
%% ------
%% A short document (1-2 pages) to sum up the job-related accomplishments
%% and experience.
%%
%% Checklist
%% ---------
%% * Contact Information
%% * Work History / Experience
%% * Education
%% * Skills
%% * Summary & Objective (optional)
%% * Hobbies & Interests (optional)
%%
%% Credits
%% -------
%% * CV vs. Resume: What is the Difference? When to Use Which?
%%   https://uptowork.com/blog/cv-vs-resume-difference
%% * How to Make a Resume: A Step-by-Step Guide (+30 Examples)
%%   https://uptowork.com/blog/how-to-make-a-resume
%% * Entry-Level Resume: Sample and Complete Guide (+20 Examples)
%%   https://uptowork.com/blog/entry-level-resume-example
%%
%% Created: 2018-04-14
%%

% English version
\documentclass{resume}

% File information shown at the footer of the last page
\fileinfo{%
  \faCopyright{} 2018--2019 Weitian LI,
  \creativecommons{by}{4.0},
  \githublink{liweitianux}{resume},
  \faCalendarAlt{} \today
}

\name{Weitian}{LI}

%\taglineicon{\faBinoculars}
%\tagline{???}
\keywords{BSD, Linux, Programming, Python, C, Shell, DevOps, SysAdmin}
%\photo{7em}{photo}

% NOTE: adjust the ordering and styles w.r.t. Chinese version
\socialinfo{
  \mobile{132-6262-0332}
  \email{liweitianux@live.com}
  \github{liweitianux} \\
  \degree{Ph.D. in Physics}
  \university{Shanghai Jiao Tong University (SJTU)}
  \birthday{1991 September}
  \home{Hunan}
}

\begin{document}
\makeheader

%======================================================================
% Summary & Objectives
%======================================================================
Highly-motivated Ph.D. in Physics (radio astronomy)
with good foundations of math and statistics.
Proficient in data modeling and analysis,
and enthusiastic about computer and network technologies.
With 10 years experience in Linux and BSD,
skilled in Shell, Python, and C programming.
Passionate about open source and share multiple projects on my
\link{https://github.com/liweitianux}{GitHub}.
Meanwhile a \link{https://www.dragonflybsd.org}{DragonFly BSD}
operating system developer and a contributor to several other
open source projects.

%======================================================================
% Competences / Skills & Languages
\sectionTitle{Competences \& Languages}{\faWrench}
%======================================================================
\begin{competences}[10em]
  \comptence{Operating Systems}{
    \icon{\faLinux} Linux (10 years),
    \icon{\faFreebsd} DragonFly BSD \& FreeBSD (7 years)
  }
  \comptence{Programming}{%
    Python, C, Shell, R, Tcl/Tk
  }
  \comptence{Tools}{%
    SSH, Git, Make, Tmux, Vi, Ansible
  }
  \comptence{Data Analysis}{%
    R, Pandas; Matplotlib, ggplot2; Keras, Scikit-learn
  }
  \comptence{Web Development}{%
    Flask, JavaScript, jQuery, Bootstrap
  }
  \comptence{\icon{\faLanguage} Languages}{
    \textbf{English} ---
      reading \& writing (good);
      listening \& speaking (conversant)
  }
\end{competences}

%======================================================================
% Education
\sectionTitle{Education}{\faGraduationCap}
%======================================================================
\begin{educations}
  \education%
    {September 2013}%
    {Shanghai Jiao Tong University}%
    {School of Physics and Astronomy}%
    {Physics}%
    {Ph.D. (candidate; to graduate in the middle of 2019)}

  \separator{0.5ex}
  \education%
    {September 2009}%
    [June 2013]%
    {Shanghai Jiao Tong University}%
    {Department of Physics and Astronomy}%
    {Applied Physics}%
    {Bachelor's Degree}
\end{educations}

%======================================================================
% Research Achievements
\sectionTitle{Research Achievements}{\faAtom}
%======================================================================
\begin{itemize}
  \item Participated in research projects:
    \enquote{Simulation of Low-Frequency Radio Sky and Separation of Weak
     Astronomical Signals} (Key Program), and
    \enquote{The X-ray Study of Galaxies and Clusters of Galaxies,
      and the Research of Cosmic Low-Frequency Radio Radiation}
    (Fund for Distinguished Young Scholars).
  \item Developed the low-frequency radio sky image simulation software:
    \link{https://github.com/liweitianux/fg21sim}{\texttt{FG21sim}}
    (Python).
  \item Developed a suite of utilities to semi-automate the \textit{Chandra}
    X-ray data analysis:
    \link{https://github.com/liweitianux/chandra-acis-analysis}{\texttt{chandra-acis-analysis}}
    (Python, Shell, Tcl).
  \item Separated the faint cosmological EoR signal along the frequency
    dimension using a Convolutional Denoising Autoencoder (CDAE).
  \item Classified the radio galaxies in the FIRST survey according to
    morphologies using a Convolutional Neutral Network (CNN).
  \item Significantly improved the modeling of radio halos,
    and integrated the instrumental effects of radio interferometers
    into the simulation pipeline.
  \item Analyzed the data of over 200 galaxy clusters observed by the
    \textit{Chandra} X-ray Observatory; improved the modeling of spectral
    background components and achieved more accurate and robust fitting
    results.
  \item Published 2 first-author and 8 co-authored SCI papers.
\end{itemize}

%======================================================================
% Computer Skills
\sectionTitle{Computer Skills}{\faCode}
%======================================================================
\begin{itemize}
  \item DragonFly BSD operating system developer:
    200+ code commits;
    help users in mailing lists and the IRC channel.
  \item Used Ansible to manage a VPS running DragonFly BSD that serves
    personal email, authoritative DNS, website, Git, IRC, etc.
  \item Built and administrated the workstations, a 4-node computer cluster,
    and network facilities for the team.
  \item Participated in building and testing the SKA high-performance
    cluster prototype (1 login node + 1 data node + 4 computing nodes)
    in Shanghai Astronomical Observatory.
  \item Designed and developed the whole website (Django, Bootstrap, jQuery)
    for \enquote{The 1st China--New Zealand Joint SKA Summer School}
    in 2014.
\end{itemize}

%======================================================================
% Internships
\sectionTitle{Internships}{\faBriefcase}
%======================================================================
\begin{experiences}
  \experience
    [April 2018]%
    {August 2018}%
    {Data engineer @ Leadvisor Technology Inc. (startup company)}%
    [\begin{itemize}
      \item Search and scrape product and advertising data from Amazon web
        (Python, Requests, BeautifulSoup).
      \item Deployed the Airflow server and database to periodically
        retrieve product sales and advertising data from Amazon.
      \item Developed the website (Flask, jQuery) to help customers to
        optimize their advertising campaigns on Amazon.
    \end{itemize}]%

  \separator{0.5em}
  \experience
    [July 2013]%
    {September 2013}%
    {Web developer @ 97 Suifang (startup company)}%
    [\begin{itemize}
      \item Developed the back-end (Django) to support user registration,
        data storage and search.
      \item Developed the front-end (jQuery, AJAX) to visualize the
        temporal variations of patient's examination indicators.
    \end{itemize}]%
\end{experiences}

%======================================================================
% Papers / Publications
\sectionTitle{Publications}{\faFileAlt}
%======================================================================
\begin{itemize}
  \small
  \item \textbf{Li,~W.}, Xu,~H., Ma,~Z., Zhu, R., Hu,~D., Zhu,~Z.,
    Gu,~J., Shan,~C., Zhu, J. \& Wu, X.-P.,
    \enquote{\it Separating the EoR Signal with a Convolutional Denoising
      Autoencoder: A Deep-learning-based Method,}
    2018, Monthly Notices of the Royal Astronomical Society
    (in revision; SCI; IF=4.96)
  \item \textbf{Li,~W.}, Xu,~H., Ma,~Z., Hu,~D., Zhu,~Z., Shan,~C.,
    Wang,~J., Gu,~J., Lian,~X., Zheng,~Q., Zhu, J. \& Wu, X.-P.,
    \enquote{\it Contribution of Radio Halos to the Foreground for
      SKA EoR Experiments,}
    2018, The Astrophysical Journal (in revision; SCI; IF=5.53)
  \item Ma,~Z., Xu,~H., Zhu,~J., Hu,~D., \textbf{Li,~W.}, Shan,~C., Zhu,~Z.,
    Lian,~X., Gu,~L., Liu,~C. \& Wu,~X.-P.,
    \enquote{\it A Machine Learning Based Morphological Classification
      of 14,251 Radio AGNs Selected from the Best--Heckman Sample,}
    2018, The Astrophysical Journal Supplement Series
    (accepted; SCI; IF=8.96)
  \item Hu,~D., Xu,~H., Kang,~X., \textbf{Li,~W.}, Zhu,~Z., Ma,~Z.,
    Shan,~C., Zhang,~Z., Gu,~L., Liu,~C. \& Wu,~X.-P.,
    \enquote{\it A Study of the Merger History of the Galaxy Group
      HCG~62 Based on X-ray Observations and SPH Simulations,}
    2017, The Astrophysical Journal
    (accepted; SCI; IF=5.53)
  \item Zheng,~Q., Johnston-Hollitt,~M., Duchesne,~S. \& \textbf{Li,~W.},
    \enquote{\it Detection of a Double Relic in the Torpedo Cluster:
      SPT-Cl J0245-5302,}
    2018, Monthly Notices of the Royal Astronomical Society, 479, 730
    (SCI; IF=4.96)
  \item Ma,~Z., Zhu,~J., \textbf{Li,~W.} \& Xu,~H.,
    \enquote{\it An Approach to Detect Cavities in X-ray Astronomical
      Images Using Granular Convolutional Neural Networks,}
    2017, IEICE Transactions on Information and System, 100(10), 2578
    (SCI; IF=0.41)
  \item Zhang,~C., Xu,~H., Zhu,~Z., \textbf{Li,~W.}, Hu,~D., Wang,~J.,
    Gu,~J., Gu,~L., Zhang,~Z., Liu,~C., Zhu,~J. \& Wu,~X.-P.,
    \enquote{\it A Chandra Study of the Image Power Spectra of 41
      Cool Core and Non-cool Core Galaxy Clusters,}
    2016, The Astrophysical Journal, 823, 116 (SCI; IF=5.53)
  \item (and 3 more co-authored SCI papers)
\end{itemize}

%======================================================================
% Awards / Scholarships / Certificates
\sectionTitle{Awards \& Certificates}{\faAward}
%======================================================================
\begin{entries}
  \entry{September 2016}%
    {Participation Award,
      The 13rd China Post-Graduate Mathematical Contest in Modeling}
  \entry{July 2014}%
    {Outstanding Teaching Assistant, College Physics}
  \entry{November 2013}%
    {Outstanding Ph.D. Student Entrance Scholarship of Shanghai Jiao Tong University}
  \entry{December 2011}%
    {National Astronomical Observatory Scholarship}
  \entry{September 2011}%
    {Network Engineer (Level 4), National Computer Rank Examination}
\end{entries}

\end{document}
