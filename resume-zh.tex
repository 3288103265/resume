%%
%% Copyright (c) 2018-2019 Weitian LI <wt@liwt.net>
%% CC BY 4.0 License
%%
%% Created: 2018-04-11
%%

% Chinese version
\documentclass[zh]{resume}

% Adjust icon size (default: same size as the text)
\iconsize{\Large}

% File information shown at the footer of the last page
\fileinfo{%
  \faCopyright{} 2019--2021, Penghui Wang \hspace{0.5em}
  \creativecommons{by}{4.0} \hspace{0.5em}
  \githublink{3288103265}{resume} \hspace{0.5em}
  \faEdit{} \today
}

% #TODO:
% [x]. 简介
% [x]. 技能与语言
% [x]. 教育背景
% [x]. 计算机技能
% [].个人项目
% [].科研成果
% [].实习经历
\name{鹏辉}{王}

\keywords{BSD, Linux, Programming, Python, C, Shell, DevOps, SysAdmin}

% \tagline{\icon{\faBinoculars}} <position-to-look-for>}
% \tagline{<current-position>}

% \photo{<height>}{<filename>}

\profile{
  \mobile{152-5657-8159}
  \wechat{abc3288103265}
  \email{wph0213@126.com}\\
  \university{中国科学技术大学}
  \degree{电子与通信工程 \textbullet 硕士}
  \home{新疆 \textbullet 北屯市}\\
  \birthday{1996-02-13}
  \github{3288103265} 
  % Custom information:
  % \icontext{<icon>}{<text>}
  % \iconlink{<icon>}{<link>}{<text>}
}



\begin{document}
\makeheader

%======================================================================
% Summary & Objectives
%======================================================================
{\onehalfspacing\hspace{2em}%
电子与通信工程专业(人工智能方向)硕士研究生,主要研究方向为计算机\\视觉(图像生成)和社交网络(流行度预测,谣言检测)。研究生阶段参与了多项\\科研项目以及相关竞赛,具有较扎实的数学和计算机基础,掌握了深度学习,图\\像处理,数据分析等技能,并且具有一定的积极实践、探索合作的精神。求职意向\\为算法工程师类。
\par}

%======================================================================
\sectionTitle{教育背景}{\faGraduationCap}
%======================================================================
\begin{educations}

  \education%
    {2019.09}%
    [2022.06]%
    {中国科学技术大学}%
    {信息科学技术学院}%
    {电子与通信工程}%
    {硕士|导师-毛震东|成绩-3.77/4.3(21/192)|荣誉-硕士一等学业奖学金}
    % {导师:毛震东}%
  \separator{0.5ex}
  \education%
    {2015.09}%
    [2019.06]%
    {中国科学技术大学}%
    {工程科学学院}%
    {精密机械与精密仪器}%
    {学士|成绩-3.17/4.3|荣誉-2017年国家励志奖学金}
\end{educations}

%======================================================================
% \sectionTitle{技能和语言}{\faWrench}
% %======================================================================
% \begin{competences}
%   % \comptence{操作系统}{%
%   %   \icon{\faLinux} Linux ,
%   %   \icon{\faFreebsd} DragonFly BSD \& FreeBSD (7 年)
%   % }
%   \comptence{编程语言}{%
%     Python, Shell, Matlab, C, C$\#$
%   }
%   \comptence{相关工具}{%
%     PyTorch, PyTorch-Lightning, OpenCV, Pandas, Matplotlib, Scikit-learn
%   }
%   \comptence{}{%
%    Docker, Git, SSH, Tmux, Vim, Latex, Jupyter, VS Code
%   }
%   \comptence{\icon{\faLanguage} 语言}{
%     \textbf{English} --- 读写,听说(日常交流),本科曾参与三个月的澳大利亚暑研项目
%   }
% \end{competences}

%======================================================================
\sectionTitle{技能与语言}{\faCogs}
%======================================================================
\begin{itemize}
  \item 计算机视觉:对图像生成框架GANs有较深入研究,对底层视觉任务(如检测与分割)有所了解,此外对于虚假图像检测、图像流行度预测、图像字幕生成等任务有简单的项目经验。
  \item 自然语言处理:有过谣言检测项目经验,对于Bert等常见的开源NLP模型与工具有简单了解并能使用。
  \item 深度学习框架:熟练使用PyTorch,包括神经网络构建,模型训练,多卡并行(DP/DDP),结果可视化等。
  \item 其他:Python(熟练),Docker, Git, ,PyTorch-Lightning, OpenCV, Pandas, Matplotlib, Scikit-learn, Latex,shell,C语言。
  \item 语言:读写,听说(日常交流),本科曾参与三个月的澳大利亚暑研项目
\end{itemize}

%======================================================================
\sectionTitle{项目经历}{\faCode}
%======================================================================
\begin{itemize}
  \item  (2020/11-现在)\textbf{多模态图像生成}:根据文本描述,场景图等,生成相应图像。通过以往方法的分析,发现在生成的图像中,某些类别物体的语义信息在特征空间距离太近,不易分辨。于是在物体级别使用对比学习,使得同类别正样本对的特征距离靠近,负样本的特征距离远离。 刷新了layout2img任务的指标,拟投论文。
  \item (2020/3-2020/5)\textbf{IEEE网络流行度预测}:根据Flickr上博文预测流行度。作为主力队员,在队伍中负责图像数据的处理以及基于TCN的算法探索。针对验证集中某类别数据的严重缺现象,提出了使用两个独立的模型分别处理缺失和未缺失两种情况,对结果进行集成。最终获得比赛第一名,并以共同一作发表论文。
  \item (2020/3-2020/5)\textbf{ICIP图像流行度预测}:预测网络图像流行度。作为队员探索了时序特征的动态平均用以提升预测结果的鲁棒性。获得第一名以及1500美金的奖金。
  \item (2019/10-2019/12)\textbf{多模态微博谣言检测}:根据文本和图像检测微博谣言。使用Bert提取文本的特征,使用VGG19提取图像特征,将两种特征结合进行谣言的预测,提升了11$\%$谣言检测的准确度,并申请专利。

\end{itemize}

%======================================================================
\sectionTitle{科研成果}{\faAtom}
%======================================================================
\begin{itemize}
  \item \textbf{A featurte generalization framework for social media popularity prediction.} ACM Meltimedia 2020. 
  Kai Wang*, \textbf{Penghui Wang*}, Xin Chen, Qishi Huang, Zhendong Mao, and Yongdong Zhang.
  \item \textbf{一种基于预训练语言模型的多模态网络谣言检测方法}。张勇东;毛震东;邓旭冉;\textbf{王鹏辉}。专利(申请)号:CN201911376275.4

\end{itemize}

% %======================================================================
% \sectionTitle{实习经历}{\faBriefcase}
% %======================================================================
% \begin{experiences}
%   \experience%
%     [2018.04]%
%     {2018.08}%
%     {数据工程师 @ 上海领脉网络科技(初创公司)}%
%     [\begin{itemize}
%       \item 从 Amazon 网页搜索并挖取商品与广告信息
%         (Python, Requests, BeautifulSoup)
%       \item 配置 Airflow 服务器和数据库等基础设施,
%         定期从 Amazon 获取产品销售与广告投放等数据
%       \item 开发网站(Flask, jQuery),帮助客户优化 Amazon 广告投放
%     \end{itemize}]

%   \separator{0.5ex}
%   \experience%
%     [2013.07]%
%     {2013.09}%
%     {网站开发 @ 97 随访(初创公司)}%
%     [\begin{itemize}
%       \item 后端开发(Django),完成用户注册、数据存储和搜索等功能
%       \item 前端开发(jQuery, AJAX),对患者各项指标随时间的变化进行可视化
%     \end{itemize}]
% \end{experiences}
% 修改意见
% 1.教育经历放第一点 ✅
% 2.计算机技能和技能与语言有重合,建议合并 
% 3.没有照片,附证件照  
% 4.项目经历方面要写得学术化一点,突出创新点 
% 5.排版方面,现在的排版貌似是从左到右铺开,太紧密不方便看,比如电话微信这些信息排列太紧凑
\end{document}
