%%
%% Copyright (c) 2018 Weitian LI
%%
%% Resume / Résumé
%% A short document (1-2 pages) to sum up the job-related accomplishments
%% and experience.
%%
%% Information Checklist:
%% * Contact Information
%% * Work History / Experience
%% * Education
%% * Skills
%% * Summary & Objective (optional)
%% * Hobbies & Interests (optional)
%%
%% References:
%% * CV vs. Resume: What is the Difference? When to Use Which?
%%   https://uptowork.com/blog/cv-vs-resume-difference
%% * How to Make a Resume: A Step-by-Step Guide (+30 Examples)
%%   https://uptowork.com/blog/how-to-make-a-resume
%% * Entry-Level Resume: Sample and Complete Guide (+20 Examples)
%%   https://uptowork.com/blog/entry-level-resume-example
%%
%% Weitian LI <liweitianux@live.com>
%% 2018-04-11
%%

% Chinese version
\documentclass[zh]{resume}

% File information shown at the footer of the last page
\fileinfo{%
  \faCopyright{} 2018 Weitian LI,
  \creativecommons{by}{4.0}, v\today
}

% position and place
\newcommand{\myposition}{数据分析师}
\newcommand{\myplace}{深圳}

\name{维天}{李}
\tagline{\myposition{} @ \myplace}
\taglineicon{\faBinoculars}
\keywords{\myposition, \myplace, Python, Linux, BSD}
\photo{7em}{photo}

\socialinfo{
  \mobile{132-6262-0332}
  \email{liweitianux@live.com}
  \github{liweitianux} \\
  \university{上海交通大学}
  \degree{物理学 \textbullet 博士(在读)} \\
  \address{上海}
  \home{湖南 \textbullet 邵阳}
  \birthday{1991-09-26}
}

\begin{document}
\makeheader

%======================================================================
% Summary & Objectives
%======================================================================
{\onehalfspacing\hspace{1.7em}
物理学专业直博 5 年级研究生,有扎实的物理、数学与统计学基础,
并且熟悉统计分析、机器学习、信号与图像处理的基本方法。
热衷计算机技术,有 10 年的 Linux 和 BSD 使用经验。
热爱自由开源精神,并积极参与 DragonFly BSD 等多个开源项目。
拥有计算机专长,能熟练使用 Python 和 R 语言,对数据分析有强烈兴趣,
真诚应聘贵公司的\emph{\myposition}职位。
\par}  % NOTE: \par is needed

%======================================================================
% Competences / Skills
\sectionTitle{技能}{\faWrench}
%======================================================================
\begin{competences}
  \comptence{操作系统}%
    {Linux(10 年), BSD(DragonFly BSD 和 FreeBSD;7 年)}
  \comptence{编程}%
    {Python, Shell, C; R, Julia}
  \comptence{数据分析}%
    {R, pandas, scikit-learn; matplotlib, ggplot2; SQL(了解)}
  \comptence{工具}%
    {正则表达式; Jupyter notebook; SSH, Git, Make; Ansible}
  \comptence{网站开发}%
    {Django, Tornado; jQuery, Bootstrap; JavaScript, HTML5}
  \comptence{排版}%
    {\LaTeX, Con\TeX{}t}
\end{competences}

%======================================================================
% Education
\sectionTitle{教育背景}{\faGraduationCap}
%======================================================================
\begin{educations}
  \education%
    {2013.09}%
    {上海交通大学}%
    {物理与天文学院}%
    {物理学}%
    {博士(在读,预计 2019 年初毕业)}

  \separator{0.7em}
  \education%
    {2009.09}%
    [2013.06]%
    {上海交通大学}%
    {物理与天文系}%
    {应用物理学}%
    {学士学位}
\end{educations}

%======================================================================
% Research Projects
\sectionTitle{科研项目}{\faCogs}
%======================================================================
\begin{projects}
  \project
    {2015.01}%
    {国家自然科学基金}%
    {重点项目}%
    {低频射电天空的高精度仿真与微弱天体辐射信号的识别}%
    {\begin{itemize}
      \item 使用 Python 开发低频射电天文模拟软件:
            \link{https://github.com/liweitianux/fg21sim}{\texttt{FG21sim}}
      \item 显著改进星系团射电晕的建模,并考虑低频干涉阵列的复杂仪器效应
      \item 量化评估射电晕对探测宇宙再电离信号的影响,并完成期刊论文
      \item 合作利用深度卷积神经网络(CNN)对 FIRST 巡天的射电星系图像
            根据形态特征进行分类
      \item 利用小波分析等算法,对 X 射线天文图像进行去噪与增强
      \item 提取 X 射线天文图像的空间和光谱信息,利用支持向量机(SVM)
            进行分类,探测点源
    \end{itemize}}%
    [Python, 机器学习, CNN, SVM, 图像处理]

  \separator{0.7em}
  \project
    {2012.07}%
    [2014.12]%
    {国家自然科学基金}%
    {杰出青年基金}%
    {星系和星系团的 X 射线研究、宇宙低频射电辐射研究}%
    {\begin{itemize}
      \item 处理 200 多个 \textit{Chandra} X 射线卫星观测的星系团数据,
            分析其图像与光谱
      \item 构建样本,搜集 SDSS 光学波段数据,研究星系团中央辐射超出
            与其中央主导星系之间的关联
      \item 编写并维护一套数据处理程序:
            \link{https://github.com/liweitianux/chandra-acis-analysis}%
              {\texttt{chandra-acis-analysis}}
    \end{itemize}}%
    [数据搜集, 数据处理, 统计分析, Python, Shell]
\end{projects}

%======================================================================
% Experience
\sectionTitle{经验}{\faBriefcase}
%======================================================================
\begin{experiences}
  \experience
    {2018.04}%
    {参加\enquote{第二届中澳 SKA 大数据工作研讨会}}%
    [\begin{itemize}
      \item 实现数据存储系统
            \link{https://github.com/ICRAR/ngas}{\texttt{NGAS}}
            与数据处理系统
            \link{https://github.com/ICRAR/daliuge}{\texttt{DALiuGE}}
            之间的数据传输功能
      \item 提升团队协作能力和学习敏捷开发方法
    \end{itemize}]%
    [数据传输, 数据存储, 敏捷开发, Python]

  \separator{0.5em}
  \experience
    {2018.03}%
    {成为 DragonFly BSD 开发者}%
    [][BSD, 开源]

%  \separator{0.5em}
%  \experience
%    {2018.03}%
%    {使用 Ansible 管理 VPS 配置,并增加个人域名的权威 DNS 服务}%
%    [][BSD, Ansible, DNS]

  \separator{0.5em}
  \experience
    {2018.02}%
    {修订\enquote{中国 SKA 科学白皮书},负责重写\enquote{低频观测设备}章节}

%  \separator{0.5em}
%  \experience
%    {2017.12}%
%    {参与配置和测试上海天文台为建设 SKA 区域科学中心而准备的
%      高性能计算集群原型机(6 节点)}

  \separator{0.5em}
  \experience
    {2017.09}%
    {撰写\enquote{中国 SKA 科学白皮书},协助完成\enquote{前景大尺度弥散源}章节}

  \separator{0.5em}
  \experience
    [2017.04]%
    {2017.08}%
    {肺部 CT 扫描图像分析}%
    [\begin{itemize}
      \item 与上海胸科医院合作,尝试通过分析 CT 图像判断肿瘤突变类型,
            帮助医生制订治疗计划
      \item 使用灰度共生矩阵(GLCM)提取图像特征,再用主成分分析(PCA)
            降维分析,发现 CT 图像提供的信息不足以有效地预测肿瘤的突变类型
    \end{itemize}]%
    [特征提取, 数据降维, PCA]

%  \separator{0.5em}
%  \experience
%    {2017.04}%
%    {配置 VPS,运行 DragonFly BSD 系统,部署个人域名邮箱、网站、
%      CalDAV/CardDAV、Git 等服务}%
%    [][BSD, Postfix, Dovecot, Nginx, Radicale, Git]

%  \separator{0.5em}
%  \experience
%    {2016.12}%
%    {搭建和管理课题组的计算机集群(4 节点),
%      用于开展流体动力学模拟,研究星系团的并合过程}%
%    [][CentOS, Slurm, 数值模拟]

%  \separator{0.5em}
%  \experience
%    {2016.11}%
%    {参加 \enquote{BSD Meetup: BSD \& Cloud} 聚会 @ 上海}%
%    [][BSD, 开源]

  \separator{0.5em}
  \experience
    {2016.09}%
    {参加\enquote{第十三届全国研究生数学建模竞赛}}%
    [\begin{itemize}
      \item 利用全基因组(GWAS)的方法定位与性状或疾病相关联的位点(SNP)和基因
      \item 使用 R 语言对样本中的位点编码与性状做 Logistic 回归分析,
            挑选出与该性状关联最强的若干位点,并进一步确定相关联的基因
    \end{itemize}]%
    [R, 数据清洗, 回归分析, 假设检验]

  \separator{0.5em}
  \experience
    [2014.04]%
    {2014.07}%
    {筹办\enquote{第一届中国--新西兰联合 SKA 暑期学校}}%
    [\begin{itemize}
      \item 设计并制作宣传海报
      \item 设计并开发网站,提供用户注册、日程管理、通知和讲义下载等功能
    \end{itemize}]%
    [设计, Django, Bootstrap, jQuery, JavaScript, MySQL]

  \separator{0.5em}
  \experience
    [2013.07]%
    {2013.09}%
    {暑期实习 @ 97 随访(初创公司)}%
    [\begin{itemize}
      \item 开发网站,用于帮助乙肝患者记录和跟踪化验报告中的各项指标
      \item 使用 Django 开发网站后端,实现用户注册、数据存储和搜索等功能
      \item 在前端使用 AJAX 技术对患者各项指标随时间的变化进行可视化
    \end{itemize}]%
    [数据库, 数据可视化, Django, AJAX]

%  \separator{0.5em}
%  \experience
%    [2010.03]%
%    {2011.09}%
%    {参与校内的开源协会}%
%    [][开源, Linux]
\end{experiences}

%======================================================================
% Languages
\sectionTitle{语言}{\faLanguage}
%======================================================================
\begin{entries}[|]
  \entry{\textbf{英语}}{%
    \begin{description}
      \item[阅读] --- 良好(顺利阅读教材和专业文献)
      \item[写作] --- 良好(撰写学术论文)
      \item[听说] --- 日常交流
    \end{description}
  }

  \separator{0.7em}
  \entry{\textbf{汉语}}{%
    \begin{description}
      \item[写作] --- 好
        (参与撰写项目申请、年度总结等;
        撰写和修订\enquote{中国 SKA 科学白皮书}章节)
      \item[表达] --- 好
        (5 学期的助教经验)
    \end{description}
  }
\end{entries}

%%======================================================================
%% Teaching Assistant
%\sectionTitle{助教}{\faUsers}
%%======================================================================
%\begin{entries}
%  \entry{2017 年春季}%
%    {宇宙与人类(通识课)}
%  \entry{2015 年秋季}%
%    {物理学引论 II (致远荣誉计划)}
%  \entry{2015 年春季}%
%    {物理学引论 I (致远荣誉计划)}
%  \entry{2014 年秋季}%
%    {物理学引论 I (致远荣誉计划)}
%  \entry{2014 年春季}%
%    {大学物理(获优秀助教)}
%\end{entries}

%======================================================================
% Awards / Scholarships / Certificates
\sectionTitle{获奖及证书}{\faTrophy}
%======================================================================
\begin{entries}
  \entry{2016.09}%
    {第十三届全国研究生数学建模竞赛 \textbullet 成功参与奖}
  \entry{2014.07}%
    {大学物理优秀助教}
  \entry{2013.11}%
    {上海交通大学优秀博士新生奖学金}
  \entry{2012.10}%
    {上海交通大学先进个人}
  \entry{2011.12}%
    {国家天文台奖学金}
  \entry{2011.09}%
    {全国计算机等级考试 \textbullet 四级网络工程师}
\end{entries}

\end{document}

%% EOF
