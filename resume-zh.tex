%%
%% Copyright (c) 2018-2019 Weitian LI <wt@liwt.net>
%% CC BY 4.0 License
%%
%% Created: 2018-04-11
%%

% Chinese version
\documentclass[zh]{resume}

% File information shown at the footer of the last page
\fileinfo{%
  \faCopyright{} 2018--2019 Weitian LI,
  \creativecommons{by}{4.0},
  \githublink{liweitianux}{resume},
  \faCalendarAlt{} \today
}

\name{维天}{李}

%\taglineicon{\faBinoculars}
%\tagline{???}
\keywords{BSD, Linux, Programming, Python, C, Shell, DevOps, SysAdmin}

\socialinfo{
  \mobile{132-6262-0332}
  \email{liweitianux@live.com}
  \github{liweitianux} \\
  \university{上海交通大学}
  \degree{物理学 \textbullet 博士}
  \birthday{1991-09-26}
  \home{湖南 \textbullet 邵阳}
}

\begin{document}
\makeheader

%======================================================================
% Summary & Objectives
%======================================================================
{\onehalfspacing\hspace{2em}%
物理学专业(射电天文方向)直博研究生,有扎实的物理、数学与统计学基础,
擅长数据建模与分析,热衷计算机和网络技术,
有 10 年的 Linux 和 BSD 使用经验,熟练掌握 Shell、Python 和 C 语言编程。
积极实践自由开源精神,
在 \link{https://github.com/liweitianux}{GitHub} 上分享多个项目,
是 \link{https://www.dragonflybsd.org}{DragonFly BSD} 操作系统的开发者,
并积极参与其他多个开源项目。
\par}

%======================================================================
\sectionTitle{技能和语言}{\faWrench}
%======================================================================
\begin{competences}
  \comptence{操作系统}{%
    \icon{\faLinux} Linux (10 年);
    \icon{\faFreebsd} DragonFly BSD \& FreeBSD (7 年)
  }
  \comptence{编程}{%
    Python, C, Shell, R, Tcl/Tk
  }
  \comptence{工具}{%
    SSH, Git, Make, Tmux, Vi, Ansible
  }
  \comptence{数据分析}{%
    R, Pandas; Matplotlib, ggplot2; Keras, Scikit-learn
  }
  \comptence{网站开发}{%
    Flask, JavaScript, jQuery, Bootstrap
  }
  \comptence{\icon{\faLanguage} 语言}{
    \textbf{英语} --- 读写(优良), 听说(日常交流)
  }
\end{competences}

%======================================================================
\sectionTitle{教育背景}{\faGraduationCap}
%======================================================================
\begin{educations}
  \education%
    {2013.09}%
    {上海交通大学}%
    {物理与天文学院}%
    {物理学}%
    {博士(直博研究生,在读,预计 2019 年中期毕业)}

  \separator{0.5ex}
  \education%
    {2009.09}%
    [2013.06]%
    {上海交通大学}%
    {物理与天文系}%
    {应用物理学}%
    {学士}
\end{educations}

%======================================================================
\sectionTitle{计算机技能}{\faCode}
%======================================================================
\begin{itemize}
  \item DragonFly BSD 操作系统开发者:
    200+ 代码提交;
    在邮件列表和 IRC 频道参与讨论和回答问题
  \item 使用 Ansible 管理 VPS,部署个人域名邮箱、权威 DNS、网站、Git、IRC 等服务
  \item 搭建并管理课题组的工作站、计算集群(4 节点)和网络设备
  \item 参与配置和测试上海天文台的 SKA 高性能计算集群原型机
    (1 管理节点 + 1 存储节点 + 4 计算节点)
  \item 设计并开发了\enquote{2014 第一届中国--新西兰联合 SKA 暑期学校}的整个网站
    (Django, Bootstrap, jQuery)
\end{itemize}

%======================================================================
\sectionTitle{科研成果}{\faAtom}
%======================================================================
\begin{itemize}
  \item 开发低频射电天空图像模拟软件:
    \link{https://github.com/liweitianux/fg21sim}{\texttt{FG21sim}}
    (Python)
  \item 开发程序实现~X~射线卫星观测数据的半自动化分析:
    \link{https://github.com/liweitianux/chandra-acis-analysis}{\texttt{chandra-acis-analysis}}
    (Python, Shell, Tcl)
  \item 利用卷积去噪自动编码器(CDAE)在频率维度分离微弱的宇宙再电离(EoR)信号
  \item 利用卷积神经网络(CNN)对 FIRST 巡天的射电星系图像根据形态特征进行分类
  \item 显著改进星系团射电晕的建模,并考虑低频干涉阵列的复杂仪器效应
  \item 改进~X~射线光谱拟合的背景成分建模,获到更准确可靠的拟合结果
  \item 发表 2 篇第一作者以及 8 篇合作者 SCI 论文
\end{itemize}

%======================================================================
\sectionTitle{实习经历}{\faBriefcase}
%======================================================================
\begin{experiences}
  \experience%
    [2018.04]%
    {2018.08}%
    {数据工程师 @ 上海领脉网络科技(初创公司)}%
    [\begin{itemize}
      \item 从 Amazon 网页搜索并挖取商品与广告信息
        (Python, Requests, BeautifulSoup)
      \item 配置 Airflow 服务器和数据库等基础设施,
        定期从 Amazon 获取产品销售与广告投放等数据
      \item 开发网站(Flask, jQuery),帮助客户优化 Amazon 广告投放
    \end{itemize}]

  \separator{0.5ex}
  \experience%
    [2013.07]%
    {2013.09}%
    {网站开发 @ 97 随访(初创公司)}%
    [\begin{itemize}
      \item 后端开发(Django),完成用户注册、数据存储和搜索等功能
      \item 前端开发(jQuery, AJAX),对患者各项指标随时间的变化进行可视化
    \end{itemize}]
\end{experiences}

\end{document}
